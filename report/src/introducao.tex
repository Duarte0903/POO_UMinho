\chapter{Introdução}

    No âmbito da Unidade Curricular de Programação Orientada aos Objetos foi-nos proposto um trabalho prático que visava a interação de vários utilizadores com uma plataforma de compra e venda de artigos.

    Tal como acontece na vida real, dentro de uma plataforma deste género um utilizador pode realizar determinadas operações, como ver o dinheiro que gastou, ou os produtos que comprou, no entanto muitas outras funcionalidades estão apenas acessíveis ao administrador do sistema, sendo que o contrário também se aplica, visto que por norma o administrador não tem a capacidade de comprar/vender artigos.

    Tendo em conta que este sistema deve ser implementado com base nos princípios da programação orientada aos objetos (encapsulamento das estruturas de dados, compatibilidade de tipos...), torna-se fundamental criar uma hierarquia de classes bem organizada que permita o crescimento sustentável do projeto, até porque futuramente devem ser inseridos novos artigos sem que isso implique uma mudança da implementação já realizada.

    Assim, ao longo deste relatório, pretendemos apresentar e justificar a hierarquia de classes que adotámos, bem como detalhes de arquitetura que nos parecem ser de extrema relevância.